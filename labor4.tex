\documentclass{article}
\usepackage[T1]{fontenc}
\usepackage{graphicx}
\usepackage{lipsum}
\usepackage{float}
\usepackage{caption}
\usepackage[hidelinks]{hyperref}

\begin{document}

\title{Exercises LaTeX - Behavior of floats and references}
\author{NONOGA Djintoba\\ Student ID: 1032249064}
\date{\today}
\maketitle

\section{Creating an image}
This picture
\begin{center}
    \includegraphics[height=3cm]{example-image}
\end{center}
is an imported PDF.

\section{Exploring image sizing keys}
\subsection{Images with different sizes}

\begin{figure}[ht]
\centering
\includegraphics[width=0.5\textwidth]{example-image-b}
\caption{Image with width=0.5\textbackslash textwidth}
\label{fig:with_text}
\end{figure}

\begin{figure}[ht]
\centering
\includegraphics[height=0.5\textheight]{example-image-c}
\caption{Image with height=0.5\textbackslash textheight}
\label{fig:height}
\end{figure}

\begin{figure}[ht]
\centering
\includegraphics[scale=0.3]{example-image}
\caption{Image with scale=0.3}
\label{fig:scale}
\end{figure}

\begin{figure}[ht]
\centering
\includegraphics[angle=50]{example-image} 
\caption{Image with 50 degree rotation}
\label{fig:angle}
\end{figure}

\subsection{Comparison \textbackslash textwidth and \textbackslash linewidth}

\begin{figure}[ht]
\centering
\includegraphics[width=0.8\textwidth]{example-image-a}
\caption{Using of \textbackslash textwidth (80\%)}
\label{fig:textwidth}
\end{figure}

\begin{figure}[ht]
\centering
\includegraphics[width=0.8\linewidth]{example-image-b}
\caption{Using of \textbackslash linewidth (80\%)}
\label{fig:linewidth}
\end{figure}

\lipsum[1-3]

\section{Testing of position specifiers}

\lipsum[4-6]

\begin{figure}[t]
\centering
\includegraphics[width=0.7\linewidth]{example-image-a}
\caption{Figure positioned at the top (t)}
\label{fig:top}
\end{figure}

\begin{figure}[b]
\centering
\includegraphics[width=0.7\linewidth]{example-image-c}
\caption{Figure positioned at the bottom}
\label{fig:bottom}
\end{figure}

\begin{figure}[ht]
\centering
\includegraphics[width=0.7\linewidth]{example-image-c}
\caption{Figure positioned here (h)}
\label{fig:here}
\end{figure}

\begin{figure}[p]
\centering
\includegraphics[width=0.8\textwidth]{example-image}
\caption{Figure on dedicated page (p)}
\label{fig:page}
\end{figure}

\begin{figure}[!ht]
\centering
\includegraphics[width=0.6\linewidth]{example-image-a}
\caption{Figure with forced placement (!ht)}
\label{fig:force}
\end{figure}

\lipsum[7-15]

\section{Sections and equations}
\subsection{Sections and subsections}
\subsection{First section}\label{sec:first}
This section has a label \texttt{sec:first}.

\subsubsection{Subsection example}\label{subsec:example}
This subsection has a label \texttt{subsec:example}.

\subsection{Numbered list}
\begin{enumerate}
\item First item\label{item:first}
\item Second item\label{item:second}
\item Third item\label{item:third}
\end{enumerate}

\subsection{Equations with labels}

\begin{equation}\label{eq:labelone}
e^{i\pi}+1=0
\end{equation}

\begin{equation}\label{eq:labeltwo}
a^2 + b^2 = c^2
\end{equation}

\section{Labels testing with float}

\subsection{Label before caption}
\begin{figure}[ht]
\centering
\includegraphics[width=0.6\linewidth]{example-image-b}
\caption{Figure with label defined before the legend}
\label{fig:before}
\end{figure}

\subsection{Label after caption}
\begin{figure}[ht]
\centering
\includegraphics[width=0.6\linewidth]{example-image-c}
\caption{Figure with label defined after the legend}
\label{fig:after}
\end{figure}

\section{Personal image}
\begin{figure}[ht]
\centering
\includegraphics[width=0.8\linewidth]{mon_image.png}
\caption{Personal image}
\label{fig:perso}
\end{figure}

\section{Cross references}
\subsection{Figure references}
Figure with \textbackslash textwidth: \ref{fig:textwidth}\\
Figure with \textbackslash linewidth: \ref{fig:linewidth}\\
Figure at top: \ref{fig:top}\\
Figure at bottom: \ref{fig:bottom}

\subsection{Sections references}
Section: \ref{sec:first}\\
Subsection: \ref{subsec:example}

\subsection{Equations references}
Equation 1: \ref{eq:labelone}\\
Equation 2: \ref{eq:labeltwo}

\subsection{Items references}
Item 1: \ref{item:first}\\
Item 2: \ref{item:second}\\
Item 3: \ref{item:third}

\section{Conclusion}

\subsection{Observed behavior}

\begin{itemize}
\item \texttt{\textbackslash textwidth} vs \texttt{\textbackslash linewidth}: In two-column mode, \texttt{\textbackslash linewidth} corresponds to the width of the current column, while \texttt{\textbackslash textwidth} corresponds to the total width of the page.

\item Position specifiers:
  \begin{itemize}
  \item \texttt{h}: Here if possible
  \item \texttt{t}: At the top of the page
  \item \texttt{b}: At the bottom of the page
  \item \texttt{p}: Dedicated page
  \item \texttt{!}: Force placement
  \end{itemize}

\item Labels and references: 2 compilations are necessary for the references to be correct.

\item Label before \texttt{\textbackslash caption}: The reference points to the section rather than the figure.

\item Label after \texttt{\textbackslash end\{equation\}}: The reference is not working properly.
\end{itemize}

\lipsum[16-20]

\end{document}